\documentclass[11pt,a4paper]{article}
\usepackage[utf8]{inputenc}
\usepackage[german]{babel}
\usepackage{amsmath}
\usepackage{amsfonts}
\usepackage{amssymb}
\usepackage{scrpage2}\pagestyle{scrheadings}
\ihead{Thomas Verweyen (759743) \\ Norman Vetter (749229)}
\setheadsepline{0.2pt}
\begin{document}
\begin{center}
\section*{ Theoretische Informatik 1 \\ Übung Blatt 1}
\end{center}
\ \\ \ \\
\subsection*{Aufgabe 1.1}
$\forall A,B,C: \underset{\Sigma}{\underbrace{A \cap (B \cup C)}} \Leftrightarrow \underset{\Sigma'}{\underbrace{(A \cap B) \cup (A \cap C)}}$\\
\ \\
$\Sigma = \{ x|(x \in B \vee x \in C) \wedge x \in A \}$\\
$\Sigma' = \{ x|(x \in A \wedge x \in B) \vee (x \in A \wedge x \in C) \}$\\
\ \\
1) $B \cup C = \{ x| x \in B \vee x \in C \}$\\
2) $A \cap (B \cup C) = \{ x|x \in A \wedge ( x \in B \vee x \in C )\}$\\
3) $A \cap B = \{x|x \in A \wedge x \in B\}$\\
\hspace*{5mm}$ A \cap C = \{x|x \in A \wedge x \in C \}$\\
\hspace*{5mm}$ (A \cap B) \cup (A \cap C) = \{x|(x \in A \wedge x \in B) \vee (x \in A \wedge x \in C) \}$\\
\hspace*{10,5cm}q.e.d.
\subsection*{Aufgabe 1.2}
add : Nat x Nat $\rightarrow$ Nat
\begin{itemize}
\item add(n,0) = n
\item add(n,succ(m)) = succ(add(n,m))
\end{itemize}
$\forall n,m \in Nat: add(n,m) = add(m,n)$\\
\subparagraph*{Nebenbeweis:}\ \\
\begin{tabular}{lll}
IA) & add(0,0) = 0\\
& add(0,1) = succ(0) = 1\\
\ \\
IS:\\
IV) & add(0,x) = x $\Rightarrow$ add(0,succ(x)) = succ(x)\\
IB) & add(0,succ(x)) = succ(add(0,x)) = succ(x)\\
&&q.e.d.\\
\end{tabular}\\
\subparagraph*{Hauptbeweis:}\ \\
\begin{tabular}{lll}
IA) & add(m,0) = m\\
	& add(m,0) = add(0,m)\\
IS:\\
IV) & add(m,0) = add(0,m) $\Rightarrow$ add (succ(m),0) = add(0,succ(m))\\
IB) & add(succ(m),0) = succ(add(m,0)) = succ(add(0,m))=add(0,succ(m))\\
\ \\
& add(m,0) = m\\
& add(0,m) = m\\
IA) & add(0,m) = add(m,0)\\
& add(m,n) = add(n,m) $\Rightarrow$ add(m,succ(n)) = add(succ(n),m))\\
& add(m, succ(n)) = succ(add(m,n)) = succ(add(n,m)) = add(succ(n),m))\\
&&q.e.d.\\
\end{tabular}
\subsection*{Aufgabe 1.3}
\paragraph*{a)}\ \\
\#:F $\rightarrow \mathbb{N}$\\
\#(A) = 0, falls $A \in T$\\
$ \# ( \neg A ) = \# (A)+1 $\\
$ \# (A \lozenge B) = \#(A) + \#(B) + 1, wobei \lozenge \in \{ \wedge, \vee, \Rightarrow, \Leftrightarrow \}$
\paragraph*{b)}\ \\
$\forall A,B \in T.P(A) \wedge \Big( P(A) \rightarrow P( \neg A) \Big ) \wedge \Big (P(A) \wedge P(B) \rightarrow P(A \vee B) \wedge P(A \wedge B) \wedge P(A \Rightarrow B) \wedge P(A \Leftrightarrow B) \Big )$\\
\paragraph*{c)}\ \\
\begin{tabular}{lll}
IA) & r(A) = 0\\
& \#(A) = 0 ,denn A $\in$ T\\
& r(A) $\leq$ \# A(w)\\
IS:\\
IV)& $r(B) \leq \#(B) \rightarrow r( \neg B) \leq \#( \neg B)$\\
IB) & $ r( \neg B) = r(B)+1 \leq \#(B)+1 = \#( \neg B)$\\
IV)& $\Big (r(A) \leq \#(A) \wedge r(B) \leq \#(B) \Big) \rightarrow r(A \lozenge B) \leq \#(A \lozenge B),wobei $\\
&$r(A \lozenge B) = max \Big (r(a),r(B) \Big )+1 \leq r(a)+r(b)+1 \leq \#(A) + \#(B)+1 = \#(A \lozenge B),$\\
&wobei $\lozenge \in \{\wedge, \vee, \Rightarrow, \Leftrightarrow \}$.\\
&&q.e.d.\\
\end{tabular}
\end{document}