\documentclass[11pt,a4paper]{article}
\usepackage[utf8]{inputenc}
\usepackage[ngerman]{babel}
\usepackage{amsmath}
\usepackage{amsfonts}
\usepackage{amssymb}
\usepackage{scrpage2}\pagestyle{scrheadings}
\usepackage[pdftex]{graphicx}
\ihead{Thomas Verweyen (759743) \\ Norman Vetter (749229)}
\setheadsepline{0.2pt}
\begin{document}
\begin{center}
\section*{ Theoretische Informatik 1 \\ Übung Blatt 8}
\end{center}
\ \\ \ \\
\subsection*{Aufgabe 8.1}
\paragraph*{a)}\ \\
\begin{tabular}{c|c|c|c|c|c|c|c|c|c|c}
&$q_0$&$q_1$&$q_2$&$q_3$&$q_4$&$q_5$&$q_6$&$q_7$&$q_8$&$q_9$\\
\hline
$q_o$&$\blacksquare$&$X_0$&$X_2$&$X_1$&$X_0$&$X_0$&$X_1$&$X_2$&$X_0$&$X_2$\\
\hline
$q_1$&$\blacksquare$&$\blacksquare$&$X_0$&$X_0$&&&$X_0$&$X_0$&&$X_0$\\
\hline
$q_2$&$\blacksquare$&$\blacksquare$&$\blacksquare$&$X_1$&$X_0$&$X_0$&$X_1$&$X_2$&$X_0$&$X_2$\\
\hline
$q_3$&$\blacksquare$&$\blacksquare$&$\blacksquare$&$\blacksquare$&$X_0$&$X_0$&$X_1$&$X_1$&$X_0$&$X_1$\\
\hline
$q_4$&$\blacksquare$&$\blacksquare$&$\blacksquare$&$\blacksquare$&$\blacksquare$&&$X_0$&$X_0$&$X_1$&$X_0$\\
\hline
$q_5$&$\blacksquare$&$\blacksquare$&$\blacksquare$&$\blacksquare$&$\blacksquare$&$\blacksquare$&$X_0$&$X_0$&&$X_0$\\
\hline
$q_6$&$\blacksquare$&$\blacksquare$&$\blacksquare$&$\blacksquare$&$\blacksquare$&$\blacksquare$&$\blacksquare$&$X_1$&$X_0$&$X_1$\\
\hline
$q_7$&$\blacksquare$&$\blacksquare$&$\blacksquare$&$\blacksquare$&$\blacksquare$&$\blacksquare$&$\blacksquare$&$\blacksquare$&$X_0$&$X_1$\\
\hline
$q_8$&$\blacksquare$&$\blacksquare$&$\blacksquare$&$\blacksquare$&$\blacksquare$&$\blacksquare$&$\blacksquare$&$\blacksquare$&$\blacksquare$&$X_0$\\
\hline
$q_9$&$\blacksquare$&$\blacksquare$&$\blacksquare$&$\blacksquare$&$\blacksquare$&$\blacksquare$&$\blacksquare$&$\blacksquare$&$\blacksquare$&$\blacksquare$\\
\end{tabular}
\ \\
\ \\
Tupel:\\
A2 = $(\{q_0,q_2,q_3,q_6,q_7,q_9,q_{1458}\},\{a,b\},\delta,q_0,\{q_{1458}\})$
\subsection*{Aufgabe 8.2}
\paragraph*{b)}\ \\
i)\\
$L_{01}=\{0^n1^n|n \in \mathbb{N}\}$ ist nicht regulär (Beweis Vorlesung slide2.5 Folie 24)\\
$L_1 := \{w_1w_2|( \exists n \in \mathbb{N} . w_1 = 0^n) \wedge (\exists n \in \mathbb{N}.w_2 = 1^n)\}$\\
Es gilt $L_1 \cap L(0^* \circ 1^*) = L_{01} \notin L_3,also ~ L_1 ~ \notin ~ L_3$\\
\ \\ 
ii)\\
$L_2 := \{(0^k2^l1^m|k,l,m \in \mathbb{N} \wedge l < 5 \wedge m=3k+1)\}$\\
$h: \{0,1,2\}^* \rightarrow \{0,1\}^* $\\
$h(0) \rightarrow 0,~h(1) \rightarrow 1,~ h(2) \rightarrow \epsilon$\\
\begin{tabular}{ll}
$h(L_2)$&$=h(L(0^k2^l1^m))$\\
&$=h(L(0^k) \circ L(2^l) \circ L(1^m))$\\
&$=h(\{0^k\}) \circ h(\{2^l\}) \circ h(\{1^m\})$\\
&$=h(\{0\}^k) \circ h(\{2\}^l) \circ h(\{1\}^m)$\\
&$=h(\{0\})^k \circ h(\{2\})^l \circ h(\{1\})^m$\\
&$=\{0\}^k \circ \{1\}^m$\\
&$=\{0^k1^m|k,m \in \mathbb{N} \wedge m=3k+1)\}$\\
\end{tabular}
\newpage
Beweis mit Pumping-Lemma:\\
Sei $n \in \mathbb{N}$ beliebig.\\

Unser Wort sei $0^n 1^{(3n+1)}$.\\
Dann ist $x = 0^i, y = 0^j ~ und ~ z = 0^{(n-i-j)} 1^{(3n+1)}$.\\
Nehmen wir $y^k$ und k=0.\\ 
Dann ist das Wort nicht in der Sprache, denn:\\
$(w=0^{(n-j)} 1^{(3n+1)} \notin L_2 )$.\\
\paragraph*{c)}\ \\
Sei $L_i = \{0^i 1^i\}$, somit besteht die Sprache $L_i$ aus dem Wort 0..0 1..1\\ (also i-mal 0, dann i-mal 1).\\
Nun sei die Vereinigung aller Sprachen mit $i \in \mathbb{N}$ die Sprache $L_{01}$ (also die Sprache $0^n 1^n$). Welche bekannterweise nicht regulär ist.

\subsection*{Aufgabe 8.3}
$L:=\{ww^R\}$  ist regulär, g.d.w. $\Sigma^* / L$ endlich ist\\
bzw. $L:=\{ww^R\}$ ist nicht regulär, g.d.w. $\Sigma^* / L = \infty$\\

Äquivalenzklassen\\
\begin{tabular}{ll}
$[\epsilon]_L $&$= \{u \in \{a,b\}^* | u \sim_L \epsilon\} = \{\epsilon\}$\\
$[a]_L $&$= \{u | \forall w. uw \in L \Leftrightarrow aw \in L\}$\\
	&$= \{u | \forall w. uw \in L \Leftrightarrow \exists v \in L.w = va\} = \{a\}$\\
$[b]_L $&$= \{u | \forall w. uw \in L \Leftrightarrow bw \in L\}$\\
	&$= \{u | \forall w.uw \in L \Leftrightarrow \exists v \in L.w = vb\} = \{b\}$\\
$[aa]_L $&$= ... = \{u | \forall w.uw \in L \Leftrightarrow \exists v \in L.w = vaa\} = \{aa\}$\\
$[ab]_L $&$= ... = \{u | \forall w.uw \in L \forall \exists v \in L.w = vba\} = \{ab\}$\\
&.\\
&.\\
&.\\
\end{tabular}
\ \\
Es gibt unendlich viele Klassen in $\{a,b\}^*/L$.\\
Somit ist die Sprache nicht regulär.
\end{document}