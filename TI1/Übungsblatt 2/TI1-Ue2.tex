\documentclass[11pt,a4paper]{article}
\usepackage[utf8]{inputenc}
\usepackage[german]{babel}
\usepackage{amsmath}
\usepackage{amsfonts}
\usepackage{amssymb}
\usepackage{scrpage2}\pagestyle{scrheadings}
\usepackage[pdftex]{graphicx}
\ihead{Thomas Verweyen (759743) \\ Norman Vetter (749229)}
\setheadsepline{0.2pt}
\begin{document}
\begin{center}
\section*{ Theoretische Informatik 1 \\ Übung Blatt 2}
\end{center}
\ \\ \ \\
\subsection*{Aufgabe 2.1}
Erstmal war uns die Aufgabenstellung nicht ganz klar, wir haben dann aber entschieden den Beweis zu führen, 
dass die in HA 2.1 beschriebene Definition der reflexiven bzw. transitiven Hülle der dritten Definition in 
PA 2.3 entspricht. Also zu beweisen, dass\\
\ \\
\begin{small}
$R^\triangle ~=~ R~ \cup ~ \{(s,s)|s \in S\} \Leftrightarrow \underset{reflexiv}{\underbrace{\forall s \in S~.~s R^\triangle s}} ~\wedge~ \underset{kleinste ~ reflexive ~ Relation ~ die ~ R ~ enthaelt}{\underbrace{\forall R' \in \{R''| R \subseteq R''\}. \big (( \forall s \in S ~.~ sR's) \wedge R^\triangle \subseteq R' \big )}}$
\end{small}\\
Auf der linken Seite steht die Definition aus der Hausaufgabe, auf der rechten Seite stehen die Bedingungen der Präsenzaufgabe.\\
\ \\
Leider ist uns der Beweis dieser Formel nicht gelungen, wir wollten aber wenigstens zeigen, dass wir es versucht haben.\\
\ \\
Der Beweis, dass $R^\triangle$ reflexiv ist:\\
$\{(s,s)|s \in S\} ~ist ~reflexiv$\\
Wenn A $\vee$ B reflexiv, dann ist auch A $\cup$ B reflexiv.\\
Daraus folgt, $R \cup \{(s,s)|s \in S\}=R^\triangle$ ist reflexiv.\\
\ \\
Der Beweis, dass $R^\triangle$, R enthält ist:\\
$R^\triangle = R \cup \{(s,s)|s \in S \} \Rightarrow R \subseteq R^\triangle$\\
\ \\
Die beiden Beweise waren noch recht offensichtlich, jetzt fehlt noch der Beweis, dass $R^\triangle$ die kleinste Relation ist, und der Beweis in die Gegenrichtung. Bei dem Beweis, dass die Relation $R^\triangle$ die kleinste reflexive Relation auf S ist, die R enthält hatten wir große Schwierigkeiten. Wir haben verschiedene Beweisverfahren ausprobiert und sind zu keinem Ergebnis gekommen.\\

\subsection*{Aufgabe 2.2}
L(A)=$\{w \in \{a,i,m,u\}^*~|w=(mia^+u)\}^*$
\subsection*{Aufgabe 2.3}
\paragraph*{a)}\ \\
$L_1~=~\{w \in \{0,1,2\}^*$ w ist die Darstellung zur Basis 3 einer Dreierpotenz\}.\\
A($L_1) = \{\{q0,q1,F\},\{0,1,2\},\delta,q0,\{q0,q1\}\}$.\\
$\delta:$\\
\includegraphics[scale=1.5]{TI23a.eps}
\paragraph*{b)}\ \\
$L_2=\{w \in \{\heartsuit,\spadesuit,\diamondsuit\}^*~|$ nach jedem $\diamondsuit$ kommt mindestens ein $\heartsuit~oder~ein~\spadesuit$\}.\\
A($L_2) = \{\{q0,q1,F\},\{\heartsuit,\spadesuit,\diamondsuit\},\delta,q0,\{q0\}\}$.\\
$\delta:$\\
\includegraphics[scale=1.5]{TI23b.eps}
\end{document}