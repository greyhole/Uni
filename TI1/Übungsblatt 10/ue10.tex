\documentclass[11pt,a4paper]{article}
\usepackage[utf8]{inputenc}
\usepackage[german]{babel}
\usepackage{amsmath}
\usepackage{amsfonts}
\usepackage{amssymb}
\usepackage{scrpage2}
\pagestyle{scrheadings}
\usepackage{qtree}
\usepackage[pdftex]{graphicx}
\ihead{Thomas Verweyen (759743) \\ Norman Vetter (749229)}
\setheadsepline{0.2pt}
\begin{document}
\begin{center}
\section*{ Theoretische Informatik 1 \\ Übung Blatt 2}
\end{center}
\ \\ \ \\
\subsection*{Aufgabe 10.2}
$G= \Big( \{Z,A,B\},\{a,b\},P,Z \Big)$
\paragraph*{a)}
$P_1= \{Z \rightarrow ZZ|a|b\}$
\begin{enumerate}
\qtreecenterfalse
\item[] $w=aaa \in L(G):$ 
\Tree [ .Z [ .{\underline{Z} Z} [ .{\underline{Z} \underline{Z}} a a ] a ] ]
\hskip 0.3in
\Tree [ .Z [ .{Z \underline{Z}} a [ .{\underline{Z} \underline{Z}} a a ] ] ]
\end{enumerate}
Lösung:\\
$P'_1=\{Z \rightarrow Za|Zb|a|b\}$
\paragraph*{b)}
$P_2=\{Z \rightarrow ABA,A \rightarrow aA| \epsilon, B \rightarrow bB|\epsilon \}$
\begin{enumerate}
\qtreecenterfalse
\item[] $w=a \in L(G):$
\Tree [ .Z [ .{AB\underline{A}} $\epsilon$ $\epsilon$ [ .{a\underline{A}} $\epsilon$ ] ] ]
\hskip 0.3in
\Tree [ .Z [ .{\underline{A}BA} [ .{a\underline{A}} $\epsilon$ ] $\epsilon$ $\epsilon$ ] ]
\end{enumerate}
Lösung:\\
$P'_2=\{Z \rightarrow Za|A, A \rightarrow bA|bB|\epsilon, B \rightarrow aB|\epsilon \}$
\paragraph*{c)}
$P_3=\{Z \rightarrow A|B, A \rightarrow aAb|ab, B \rightarrow abB| \epsilon\}$
\begin{enumerate}
\qtreecenterfalse
\item[] $w=ab \in L(G):$
\Tree [ .Z [ .A ab ] ]
\hskip 0.3in
\Tree [ .Z [ .B [ .{ab\underline{B}} $\epsilon$ ] ] ] 
\end{enumerate}
Lösung:\\
$P'_3=\{Z \rightarrow A|B, A \rightarrow aAb|aabb, B \rightarrow abB| \epsilon\}$
\subsection*{Aufgabe 10.3}
\paragraph*{a)}
Für $G_1= (\{S\},\{a,b,c\},P,S)$ mit\\
$P=\{S \rightarrow aSbScS|aScSbS|bSaScS|bScSaS|cSaSbS|cSbSaS|\epsilon \}$\\
soll gelten: $L(G_1)=\{w \in \{a,b,c\}| |w|_a=|w|_b=|w|_c \}$.\\
\ \\
Beweis:\\
Für jede Produktion $(\Gamma^+ \rightarrow \Gamma^*)\in P$ gilt: $w \rightarrow z \wedge |w|_a=|w|_b=|w|_c \Leftrightarrow |z|_a=|z|_b=|z|_c$.\\
Da sich somit je Produktionssubstitution, die Anzahl der Terminale a,b und c oder Nichtterminale (S) gleichmäßig (oder im Falle von $S \rightarrow \epsilon$ garnicht) erhöht. Ist die Bedingung: $L(G_1)=\{w \in \{a,b,c\}| |w|_a=|w|_b=|w|_c \}$ wahr und gilt.\\

\paragraph*{b)}\ \\
$L=\{a^nb^{n+m}c^m|n,m \in \mathbb{N}\}$\\
$G2=\Big(\{S,A,C\},\{a,b,c\},P,S \Big)$\\
$P=\{S \rightarrow AC, A \rightarrow aAb|\epsilon, C \rightarrow bCc|\epsilon\}$\\
$S \rightarrow^* w \Leftrightarrow \exists n,m \in \mathbb{N}.w=a^nb^{n+m}c^m$\\
\ \\
\begin{tabular}{ll}
IA)&$S \rightarrow^3 w \Leftrightarrow \Big( (S \rightarrow AC \rightarrow \epsilon C \rightarrow \epsilon \epsilon) \vee (S \rightarrow AC \rightarrow A\epsilon \rightarrow \epsilon \epsilon) \Big) \Leftrightarrow \exists n,m \in \mathbb{N}$\\
IS:\\
IV)&$S \rightarrow^l w \Leftrightarrow \exists n,m \in \mathbb{N}.w=a^nb^{n+m}c^m$\\
&$S \rightarrow^{l+1} v \Leftrightarrow (S \rightarrow AC \rightarrow aAbC \rightarrow^l v) \vee (S \rightarrow AC \rightarrow AbCc \rightarrow^l v)$\\
&$\Leftrightarrow \exists w \in \{a,b,c\}^*. S \rightarrow^l w \wedge (|v|_a=|w|_a + 1 \vee |v|_c=|w|_c + 1) \wedge |v|b=|w|_b + 1$\\
\end{tabular}
q.e.d.
\end{document}