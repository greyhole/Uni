\documentclass[11pt,a4paper]{article}
\usepackage[utf8]{inputenc}
\usepackage[T1]{fontenc}
\usepackage[german]{babel}
\usepackage{amsmath}
\usepackage{amsfonts}
\usepackage{amssymb}
\usepackage{tabu}
\usepackage{scrpage2}
\pagestyle{scrheadings}
\usepackage{qtree}
\usepackage{paralist}
\usepackage[pdftex]{graphicx}
\ihead{Thomas Verweyen (759743), Norman Vetter (749229)\\ Lukas Sprinck (770616)}
\setheadsepline{0.2pt}
\begin{document}
\begin{center}
\section*{ Theoretische Informatik 1 \\ Übung Blatt 13}
\end{center}
\section*{Aufgabe 13.1}
\subsection*{a)}
$G=(\{S,A,B,C,D,E,F\},\{a,b,c\},P,S)$\\
$\begin{tabu}{ll}
P=&\{S \rightarrow aaAac | cbBb | \epsilon,\\
&A \rightarrow SF | C | aab,\\
&B \rightarrow aB | Bbba | BD | cbE,\\
&C \rightarrow \epsilon | S,\\
&D\rightarrow a | Ebc,\\
&E \rightarrow b | Bb |BC,\\
&F \rightarrow ACA | CccE\}
\end{tabu}$


\paragraph*{1. $\epsilon$-Übergänge eliminieren:}
\ \\
$G_1=(\{S,A,B,C,D,E,F\},\{a,b,c\},P,S)$\\
$\begin{tabu}{ll}
P_1=&\{S \rightarrow aaAac | cbBb | aaac,\\
&A \rightarrow SF | C | aab | F,\\
&B \rightarrow aB | Bbba | BD | cbE,\\
&C \rightarrow S,\\
&D\rightarrow a | Ebc,\\
&E \rightarrow b | Bb |BC | B,\\
&F \rightarrow A | C | AC | CA | AA | ACA | CccE | ccE\}
\end{tabu}$

\paragraph*{2. Einheitproduktionen eliminieren:}
\ \\
Einheitsproduktionen:\\
(A,F),(A,C),(C,S),(A,S),(E,B),(F,A),(F,C),(F,S)\\
\ \\
$G_2=(\{S,A,B,C,D,E,F\},\{a,b,c\},P,S)$\\
$\begin{tabu}{ll}
P_2=&\{S \rightarrow aaAac | cbBb | aaac,\\
&A \rightarrow SF | AC | CA | AA | ACA | CccE | ccE | aab | aaAac | cbBb | aaac,\\
&B \rightarrow aB | Bbba | BD | cbE,\\
&C \rightarrow aaAac | cbBb | aaac,\\
&D \rightarrow a | Ebc,\\
&E \rightarrow b | Bb |BC | aB | Bbba | BD | cbE,\\
&F \rightarrow SF | aab | aaAac | cbBb | aaac | AC | CA | AA | ACA | CccE | ccE\}\\
\end{tabu}$

\paragraph*{3. Erreichbarkeit und Produktion testen}
\ \\
Erreichbar: S,A,B,a,b,c,C,D,E,F\\
Priduzierend: a,b,c,S,A,B,C,D,E,F\\
$\Rightarrow$ es wird nichts geändert.

\paragraph*{4. Terminale auflösen}
\ \\
$G_3=(\{S,A,B,C,D,E,F,X_a,X_b,X_c\},\{a,b,c\},P,S)$\\
$\begin{tabu}{lll}
P_3=&\{S \rightarrow& X_aX_aAX_aX_c | X_cX_bBX_b | X_aX_aX_aX_c,\\
&A \rightarrow& SF | AC | CA | AA | ACA | CX_cX_cE | X_cX_cE | X_aX_aX_b | X_aX_aAX_aX_c |\\
 &&X_cX_bBX_b | X_aX_aX_aX_c,\\
&B \rightarrow& X_aB | BX_bX_bX_a | BD | X_cX_bE,\\
&C \rightarrow& X_aX_aAX_aX_c | X_cX_bBX_b | X_aX_aX_aX_c,\\
&D \rightarrow& X_a | EX_bX_c,\\
&E \rightarrow& X_b | BX_b |BC | X_aB | BX_bX_bX_a | BD | X_cX_bE,\\
&F \rightarrow& SF | X_aX_aX_b | X_aX_aAX_aX_c | X_cX_bBX_b | X_aX_aX_aX_c | AC | CA | AA |\\
&&ACA | CX_cX_cE | X_cX_cE,\\
&X_a \rightarrow& a,\\
&X_b \rightarrow& b,\\
&X_c \rightarrow& c\}
\end{tabu}$

\paragraph*{5. Aufspalten von Produktionen}
\ \\
$G_4=(\{S,A,B,C,D,E,F,Y_{aAac},Y_{Aac},Y_{ac},Y_{bBb},Y_{Bb},Y_{aac},Y_{ba},Y_{ccE},Y_{ab},$\\
$Y_{bba},Y_{ac},Y_{cE},Y_{CA},Y_{bE},Y_{bc}\},\{a,b,c\},P,S)$\\
$\begin{tabu}{ll}
P_4=&\{S \rightarrow X_aY_{aAac} | X_cY_{bBb} | X_aY_{aac},\\
&A \rightarrow SF | AC | CA | AA | AY_{CA} | CY_{ccE} | X_cY_{cE} | X_aY_{ab} | X_aY_{aAac} | X_cY_{bBb} | X_aY_{aac},\\
&B \rightarrow X_aB | BY_{bba} | BD | X_cY_{bE},\\
&C \rightarrow X_aY_{aAac} | X_cY_{bBb} | X_aY_{aac},\\
&D \rightarrow X_a | EY_{bc},\\
&E \rightarrow X_b | Bb |BC | X_aB | BY_{bba} | BD | X_cY_{bE},\\
&F \rightarrow SF | X_aY_{ab} | X_aY_{aAac} | X_cY_{bBb} | X_aY_{aac} | AC | CA | AA | ACA | CY_{ccE} | X_cY_{cE}\}\\
&\begin{tabu}{lll}
&Y_{aAac} \rightarrow X_aY_{Aac},&Y_{Aac} \rightarrow Y_{ac},\\
&Y_{bBb} \rightarrow X_bY_{Bb},&Y_{Bb} \rightarrow BX_b,\\
&Y_{aac} \rightarrow X_aY_{ac},&Y_{ba} \rightarrow X_bX_a,\\
&Y_{ccE} \rightarrow X_cY_{cE},&Y_{ab} \rightarrow X_aX_b,\\
&Y_{bba} \rightarrow X_bY_{ba},&Y_{ac} \rightarrow X_aX_c\\
&Y_{cE} \rightarrow X_cE,&X_a \rightarrow a,\\
&Y_{CA} \rightarrow CA,&X_b \rightarrow b,\\
&Y_{bE} \rightarrow X_bE,&X_c \rightarrow c\}\\
&Y_{bc} \rightarrow X_bX_c,
\end{tabu}
\end{tabu}$
\newpage
\paragraph{b)}
\ \\
$w_1=aaaabac$\\
$\begin{tabu}{|c|c|c|c|c|c|c|}
a&a&a&a&b&a&c\\
\hline
\{D,X_a\}& \{D,X_a\}& \{D,X_a\}&\{D,X_a\}&\{E,X_b\}&\{D,X_a\}&\{X_c\}\\
\cline{1-7}
\blacksquare&\blacksquare&\blacksquare&\{Y_{ab}\}&\{Y_{ba}\}&\{Y_{ac}\}\\
\cline{1-6}
\blacksquare&\blacksquare&\{A,F\}&\blacksquare&\blacksquare\\
\cline{1-5}
\blacksquare&\blacksquare&\blacksquare&\blacksquare\\
\cline{1-4}
\blacksquare&\blacksquare&\{Y_{Aac}\}\\
\cline{1-3}
\blacksquare&\{Y_{aAac}\}\\
\cline{1-2}
\{S\}\\
\cline{1-1}
\end{tabu}$
\ \\
$\Rightarrow ~ w_1 \in L(G_4)$\\
\ \\
$w_2=cbacbbb$\\
$\begin{tabu}{|c|c|c|c|c|c|c|}
c&b&a&c&b&b&b\\
\cline{1-7}
\{X_c\}&\{X_b,E\}&\{X_a,D\}&\{X_c\}&\{X_b,E\}&\{X_b,E\}&\{X_b,E\}\\
\cline{1-7}
\{Y_{cE}\}&\{Y_{ba}\}&\{Y_{ac}\}&\{Y_{cE}\}&\{Y_{bE}\}&\{Y_{bE}\}\\
\cline{1-6}
\blacksquare&\blacksquare&\blacksquare&\{B,E\}&\blacksquare\\
\cline{1-5}
\blacksquare&\blacksquare&\{B,E\}&\{Y_{Bb}\}\\
\cline{1-4}
\blacksquare&\{Y_{bE}\}&\{E,Y_{Bb}\}\\
\cline{1-3}
\{B\}&\{Y_{bE},Y_{bBb}\}\\
\cline{1-2}
\{S\}\\
\cline{1-1}
\end{tabu}$
\ \\
$\Rightarrow ~ w_2 \in L(G_4)$\\

\section*{Aufgabe 13.2}
Zeige das $L=\{0^m1^{m*n}2^n|n,m \in \mathbb{N}\} \notin \mathcal{L}_2$.\\
Annahme: $L \in \mathcal{L}_2$.\\
Sei n' beliebig aber fest. Wir wählen $z=0^m1^{m*p}2^p$ mit $m,p \in \mathbb{N}.m,p \geq n'$.\\
Sei $u,v,w,x,y \in \Gamma^*$ bieliebig mit z=uvwxy,\\
(1) $v\circ x \neq \epsilon$ und\\
(2)$|vwx| \leq n'$\\
\ \\

$\begin{tabu}{ll}
1.Fall)\\
&(v \circ x) \in L(0^+ + 1^+ + 2^+):\\
&u=0^a,v=0^b,w=0^c,x=0^d,y=0^{m-a-b-c-d}1^{m*p}2^p\\
&$für$~ b,d \neq 0 ~und~ b+c+d \leq n'.\\
&$Wir wählen: i=2.$\\
&\Rightarrow uv^2wx^2y=0^{m+b+d}1^{m*p}2^p \notin L.\\
&da ~ |vwx| \leq n' \leq m,p~$gilt dies ebenfalls für$~ vx=1^+ ~und~ vx=2^+.\\
\ \\
2.Fall)\\
&(v \circ x) \in L(0^+1^+ + 1^+2^+):\\
&|vx|_0 * p \neq |vx|_1$ gilt immer, denn $ p > n’ \geq |vwx| \geq |vx| \\
&a := |vx|_0, b := |vx|_1 \\
I. &a * p = b \\
II.&a + b \leq n \\
III.&p > n \\
$I in II$.& a + a * p \leq n \\
&\Leftrightarrow a * (p + 1) \leq n \\
&\Leftrightarrow a = 0 \\
&\Rightarrow b = 0 \\
&\Rightarrow $ (1) ist verletzt $\\
\ \\

&$Weiterführend sei i = 0$ \\
&\Rightarrow z$ liegt nicht mehr in der Sprache$ \\
&$Der Beweis ist für $|vx|=1^+2^+$analog. $\\
\end{tabu}$
Da wir in allen Fällen zu Wiedersprüchen gelangen ist $L \notin \mathcal{L}_2$.

\section*{Aufgabe 13.3}
$G = (V,T,P,S), P \subset \{ A \rightarrow uBv \lor A \rightarrow \epsilon | A,B \in V, u,v \in T^*\}$ \\
\\
\textit{(Dabei sei $S,A \in V, u,v,w,x,y,z \in T^*, m,i \in \mathbb N$)} \\
In jeder Ableitung $S \rightarrow^k z$ mit $k = |V| + 1$ kommt mindestens ein Nichtterminal A mindestens zweimal vor. Diese Wiederholung von A erzeugt dabei mindestens ein Terminal: \\
$A \rightarrow^* vAx \Rightarrow |vx| \geq 1$ \\
\\
Sei $S \rightarrow^* uAy \rightarrow^m uvAxy \rightarrow^* z$ eine Ableitung eines Wortes $z \in L(G)$ mit $|z| \geq n$. \\
Dann existiert auch ein $w' \in L(G)$, mit einer Ableitung: \\
$S \rightarrow^* uAy \rightarrow^{m^i} uv^iAx^iy \rightarrow^* z'$
\\
$\Leftrightarrow$
\\
Ist L eine lineare Sprache, so existiert eine Konstante $n \in \mathbb N^+$, \\
{\centering\tiny{(Dabei ist n $>$ Summe der L\"angen aller Terminalteilw\"orter, die im Rumpf einer Regel von P stehen)}}\\
sodass jedes Wort $z \in L$ mit L\"ange $|z| \geq n$ zerlegt werden kann in z = uvwxy \\
{\centering\tiny{($S \rightarrow^* uAy \rightarrow^m uvAxy \rightarrow uvwxy$)}}\\
mit den Eigenschaften \\
1. $|uvxy| \leq n$ \\
{\centering\tiny{(, denn uvxy ist bei der Abarbeitung ohne Wiederholung entstanden)}}\\
2. $|vx| \geq 1$ \\
{\centering\tiny{(, denn vx ist das Teilwort, dass bei der Wiederholung von A entsteht)}}\\
3. $\forall i \geq 0. uv^iwx^iz \in L$
\ \\
\ \\
\ \\
\paragraph{Anmerkung wegen Gruppenwechsel}
\ \\
Aufgrund eines Gruppenwechsels setzt sich unsere Gruppe nun aus den in der Kopfzeile benannten Personen zusammen.
Neu hinzu kam dabei Lukas Sprinck.
Vorherige Übungsabgaben erfolgten bei:
\begin{itemize}
\item Thomas Verweyen (759743):G4 (Tina Beigel),
\item Norman Vetter (749229):G4 (Tina Beigel),
\item Lukas Sprinck (770616):G3 (Thorsten Alten).
\end{itemize}
\end{document}