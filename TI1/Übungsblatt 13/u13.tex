\documentclass[11pt,a4paper]{article}
\usepackage[utf8]{inputenc}
\usepackage[T1]{fontenc}
\usepackage[german]{babel}
\usepackage{amsmath}
\usepackage{amsfonts}
\usepackage{amssymb}
\usepackage{tabu}
\usepackage{scrpage2}
\pagestyle{scrheadings}
\usepackage{qtree}
\usepackage[pdftex]{graphicx}
\ihead{Thomas Verweyen (759743) \\ Norman Vetter (749229)}
\setheadsepline{0.2pt}
\begin{document}
\begin{center}
\section*{ Theoretische Informatik 1 \\ Übung Blatt 13}
\end{center}
\subsection*{Aufgabe 13.1}
\paragraph*{a}
\ \\
$G=(\{S,A,B,C,D,E,F\},\{a,b,c\},P,S)$\\
$\begin{tabu}{lllll}
P=&\{S \rightarrow aaAac &| cbBb &| \epsilon,\\
&A \rightarrow SF &| C &| aab,\\
&B \rightarrow aB &| Bbba &| BD &| cbE,\\
&C \rightarrow \epsilon &| S,\\
&D\rightarrow a &| Ebc,\\
&E \rightarrow b &| Bb &|BC,\\
&F \rightarrow ACA &| CccE\}
\end{tabu}$
\ \\
\ \\
1) $\epsilon$-Übergänge eliminieren:\\
$\begin{tabu}{llllllllllll}
P=&\{S \rightarrow aaAac | cbBb | aaac,\\
&A \rightarrow SF | C | aab | F,\\
&B \rightarrow aB | Bbba | BD | cbE,\\
&C \rightarrow S,\\
&D\rightarrow a | Ebc,\\
&E \rightarrow b | Bb |BC | B,\\
&F \rightarrow A | C | AC | CA | AA | ACA | CccE | ccE\}
\end{tabu}$
\ \\
\ \\
2) Einheitproduktionen eliminieren:\\
Einheitsproduktionen:\\
(A,F),(A,C),(C,S),(A,S),(E,B),(F,A),(F,C),(F,S)\\
$\begin{tabu}{ll}
P=&\{S \rightarrow aaAac | cbBb | aaac,\\
&A \rightarrow SF | AC | CA | AA | ACA | CccE | ccE | aab | aaAac | cbBb | aaac,\\
&B \rightarrow aB | Bbba | BD | cbE,\\
&C \rightarrow aaAac | cbBb | aaac,\\
&D \rightarrow a | Ebc,\\
&E \rightarrow b | Bb |BC | aB | Bbba | BD | cbE,\\
&F \rightarrow SF | aab | aaAac | cbBb | aaac | AC | CA | AA | ACA | CccE | ccE\}\\
\end{tabu}$
\ \\
\ \\
4) Terminale auflösen:\\
$\begin{tabu}{ll}
P=&\{S \rightarrow X_aX_aAX_ac | cX_bBX_b | X_aX_aX_ac,\\
&A \rightarrow SF | AC | CA | AA | ACA | CccE | ccE | X_aX_aX_b | X_aX_aAX_ac | cX_bBX_b | X_aX_aX_ac,\\
&B \rightarrow X_aB | BX_bX_bX_a | BD | cX_bE,\\
&C \rightarrow X_aX_aAX_ac | cX_bBX_b | X_aX_aX_ac,\\
&D \rightarrow X_a | EX_bc,\\
&E \rightarrow X_b | BX_b |BC | X_aB | BX_bX_bX_a | BD | cX_bE,\\
&F \rightarrow SF | X_aX_aX_b | X_aX_aAX_ac | cX_bBX_b | X_aX_aX_ac | AC | CA | AA | ACA | CccE | ccE,\\
&X_a \rightarrow a,\\
&X_b \rightarrow b\}
\end{tabu}$
\ \\
\ \\
5) ...
$\begin{tabu}{ll}
P=&\{S \rightarrow X_aY_{aAac} | cbBb | aaac,\\
&A \rightarrow SF | AC | CA | AA | ACA | CccE | ccE | aab | aaAac | cbBb | aaac,\\
&B \rightarrow aB | Bbba | BD | cbE,\\
&C \rightarrow aaAac | cbBb | aaac,\\
&D \rightarrow a | Ebc,\\
&E \rightarrow b | Bb |BC | aB | Bbba | BD | cbE,\\
&F \rightarrow SF | aab | aaAac | cbBb | aaac | AC | CA | AA | ACA | CccE | ccE\}\\
&X_a \rightarrow a,\\
&X_b \rightarrow b\}
\end{tabu}$

\end{document}