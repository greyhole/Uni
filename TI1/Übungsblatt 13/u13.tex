\documentclass[11pt,a4paper]{article}
\usepackage[utf8]{inputenc}
\usepackage[T1]{fontenc}
\usepackage[german]{babel}
\usepackage{amsmath}
\usepackage{amsfonts}
\usepackage{amssymb}
\usepackage{tabu}
\usepackage{scrpage2}
\pagestyle{scrheadings}
\usepackage{qtree}
\usepackage[pdftex]{graphicx}
\ihead{Thomas Verweyen (759743), Norman Vetter (749229)\\ Lukas Sprinck (770616)}
\setheadsepline{0.2pt}
\begin{document}
\begin{center}
\section*{ Theoretische Informatik 1 \\ Übung Blatt 13}
\end{center}
\subsection*{Aufgabe 13.1}
\paragraph*{a)}
\ \\
$G=(\{S,A,B,C,D,E,F\},\{a,b,c\},P,S)$\\
$\begin{tabu}{ll}
P=&\{S \rightarrow aaAac | cbBb | \epsilon,\\
&A \rightarrow SF | C | aab,\\
&B \rightarrow aB | Bbba | BD | cbE,\\
&C \rightarrow \epsilon | S,\\
&D\rightarrow a | Ebc,\\
&E \rightarrow b | Bb |BC,\\
&F \rightarrow ACA | CccE\}
\end{tabu}$
\ \\
\ \\
\subparagraph{1.$\epsilon$-Übergänge eliminieren:}
\ \\
$G_1=(\{S,A,B,C,D,E,F\},\{a,b,c\},P,S)$\\
$\begin{tabu}{ll}
P_1=&\{S \rightarrow aaAac | cbBb | aaac,\\
&A \rightarrow SF | C | aab | F,\\
&B \rightarrow aB | Bbba | BD | cbE,\\
&C \rightarrow S,\\
&D\rightarrow a | Ebc,\\
&E \rightarrow b | Bb |BC | B,\\
&F \rightarrow A | C | AC | CA | AA | ACA | CccE | ccE\}
\end{tabu}$
\ \\
\ \\
\subparagraph{2. Einheitproduktionen eliminieren:}
\ \\
Einheitsproduktionen:\\
(A,F),(A,C),(C,S),(A,S),(E,B),(F,A),(F,C),(F,S)\\
$G_2=(\{S,A,B,C,D,E,F\},\{a,b,c\},P,S)$\\
$\begin{tabu}{ll}
P_2=&\{S \rightarrow aaAac | cbBb | aaac,\\
&A \rightarrow SF | AC | CA | AA | ACA | CccE | ccE | aab | aaAac | cbBb | aaac,\\
&B \rightarrow aB | Bbba | BD | cbE,\\
&C \rightarrow aaAac | cbBb | aaac,\\
&D \rightarrow a | Ebc,\\
&E \rightarrow b | Bb |BC | aB | Bbba | BD | cbE,\\
&F \rightarrow SF | aab | aaAac | cbBb | aaac | AC | CA | AA | ACA | CccE | ccE\}\\
\end{tabu}$
\ \\
\ \\
\subparagraph{3. Erreichbarkeit und Produktion testen}
\ \\
Erreichbar: S,A,B,a,b,c,C,D,E,F\\
Priduzierend: a,b,c,S,A,B,C,D,E,F\\
$\Rightarrow$ es wird nichts geändert.

\subparagraph{4. Terminale auflösen}
\ \\
$G_3=(\{S,A,B,C,D,E,F,X_a,X_b,X_c\},\{a,b,c\},P,S)$\\
$\begin{tabu}{lll}
P_3=&\{S \rightarrow& X_aX_aAX_aX_c | X_cX_bBX_b | X_aX_aX_aX_c,\\
&A \rightarrow& SF | AC | CA | AA | ACA | CX_cX_cE | X_cX_cE | X_aX_aX_b | X_aX_aAX_aX_c |\\
 &&X_cX_bBX_b | X_aX_aX_aX_c,\\
&B \rightarrow& X_aB | BX_bX_bX_a | BD | X_cX_bE,\\
&C \rightarrow& X_aX_aAX_aX_c | X_cX_bBX_b | X_aX_aX_aX_c,\\
&D \rightarrow& X_a | EX_bX_c,\\
&E \rightarrow& X_b | BX_b |BC | X_aB | BX_bX_bX_a | BD | X_cX_bE,\\
&F \rightarrow& SF | X_aX_aX_b | X_aX_aAX_aX_c | X_cX_bBX_b | X_aX_aX_aX_c | AC | CA | AA |\\
&&ACA | CX_cX_cE | X_cX_cE,\\
&X_a \rightarrow& a,\\
&X_b \rightarrow& b,\\
&X_c \rightarrow& c\}
\end{tabu}$
\ \\
\ \\
\subparagraph{5. Aufspalten von Produktionen}
\ \\
$G_4=(\{S,A,B,C,D,E,F,Y_{aAac},Y_{Aac},Y_{ac},Y_{bBb},Y_{Bb},Y_{aac},Y_{ba},Y_{ccE},Y_{ab},$\\
$Y_{bba},Y_{ac},Y_{cE},Y_{CA},Y_{bE},Y_{bc}\},\{a,b,c\},P,S)$\\
$\begin{tabu}{ll}
P_4=&\{S \rightarrow X_aY_{aAac} | X_cY_{bBb} | X_aY_{aac},\\
&A \rightarrow SF | AC | CA | AA | AY_{CA} | CY_{ccE} | X_cY_{cE} | X_aY_{ab} | X_aY_{aAac} | X_cY_{bBb} | X_aY_{aac},\\
&B \rightarrow X_aB | BY_{bba} | BD | X_cY_{bE},\\
&C \rightarrow X_aY_{aAac} | X_cY_{bBb} | X_aY_{aac},\\
&D \rightarrow X_a | EY_{bc},\\
&E \rightarrow X_b | Bb |BC | X_aB | BY_{bba} | BD | X_cY_{bE},\\
&F \rightarrow SF | X_aY_{ab} | X_aY_{aAac} | X_cY_{bBb} | X_aY_{aac} | AC | CA | AA | ACA | CY_{ccE} | X_cY_{cE}\}\\
&\begin{tabu}{lll}
&Y_{aAac} \rightarrow X_aY_{Aac},&Y_{Aac} \rightarrow Y_{ac},\\
&Y_{bBb} \rightarrow X_bY_{Bb},&Y_{Bb} \rightarrow BX_b,\\
&Y_{aac} \rightarrow X_aY_{ac},&Y_{ba} \rightarrow X_bX_a,\\
&Y_{ccE} \rightarrow X_cY_{cE},&Y_{ab} \rightarrow X_aX_b,\\
&Y_{bba} \rightarrow X_bY_{ba},&Y_{ac} \rightarrow X_aX_c\\
&Y_{cE} \rightarrow X_cE,&X_a \rightarrow a,\\
&Y_{CA} \rightarrow CA,&X_b \rightarrow b,\\
&Y_{bE} \rightarrow X_bE,&X_c \rightarrow c\}\\
&Y_{bc} \rightarrow X_bX_c,
\end{tabu}
\end{tabu}$
\paragraph{b)}
\ \\
$\begin{tabu}{|c|c|c|c|c|c|c|}
a&a&a&a&b&a&c\\
\hline
\{D,X_a\}&\{D,X_a\}&\{D,X_a\}&\{D,X_a\}&\{E,X_b\}&\{D,X_a\}&\{X_c\}\\
\{\}&
\end{tabu}$
\end{document}