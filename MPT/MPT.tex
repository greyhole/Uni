\documentclass{beamer}

\usetheme{Antibes}

\usepackage[ngerman]{babel}
\usepackage[utf8x]{inputenc}
\usepackage{totpages}
\usepackage{graphicx}
\usepackage[export]{adjustbox}
\usepackage{multicol}

\setbeamercovered{transparent}
\beamertemplatenavigationsymbolsempty
%\setbeamertemplate{footline}[frame number]
\setbeamertemplate{footline}
  {%
    \begin{beamercolorbox}[ht=2.5ex,dp=1.125ex,%
      leftskip=.3cm,rightskip=.3cm plus1fil]{upper separation line foot}
       \hfill Folie~\thepage / \ref{TotPages}
    \end{beamercolorbox}
    \begin{beamercolorbox}[ht=2.5ex,dp=1.125ex,%
      leftskip=.3cm,rightskip=.3cm plus1fil]{author in head/foot}%
      \leavevmode{\usebeamerfont{author in head/foot}\insertshortauthor}%
      \hfill%
      {\usebeamerfont{institute in head/foot}\usebeamercolor[fg]{institute in 						head/foot}\insertshortinstitute}%
    \end{beamercolorbox}%
    \begin{beamercolorbox}[colsep=1.5pt]{lower separation line foot}	
    \end{beamercolorbox}
  }
  
\title{Microprozessortechnik - MSP430}
\author{N. Vetter}
\date{\today}

\institute[Universität Potsdam]{
	Institut für Informatik
}

\begin{document}

\begin{frame}

\titlepage

\end{frame}

\section{Übersicht}
\begin{frame}
\frametitle{Gliederung}
\begin{multicols}{2}
\tableofcontents
\end{multicols}
\end{frame}

\section{Allgemeines}
\subsection{Vorstellung}
\begin{frame}
MSP430:
\begin{itemize}
\item 16-Bit-RISC-Mikrocontroller
\item Firma: Texas Instruments
\item entwickelt in Freising, Deutschland
\item Sprachen: Assempler, C
\item für möglichst niedrigen Stromverbrauch optimiert
\end{itemize}
\ \\
\ \\
Anwendung:
\begin{itemize}
\item Wärmezähler
\item Blutdruckmessgeräte
\item Tauchcomputer
\end{itemize}
\end{frame}
\section{Architektur}
\subsection{Eigenschaften}
\begin{frame}
\begin{itemize}
\item klassische Von-Neumann-Architektur
\item Datenbus ( 16-Bit )
\item Adressbus ( 16-Bit )
\item 16 16-Bit Register
\item unterschiedliche Speichergrößen möglich
\item unterschiedlichste Peripherie möglich
\item Low-Power-Mode
\item Hardwaredesign nach Software Bedürfnissen $\rightarrow$ sehr effizienter und kompakter Code durch Compiler
\end{itemize}
\end{frame}
\subsection{Grafik}
\begin{frame}
\begin{figure}[b]
\includegraphics[scale=0.63]{MSP430_block_dia.jpg}
\end{figure}
\end{frame}
\section{Memory}
\subsection{Anordnung}
\begin{frame}
\begin{figure}[t]
\includegraphics[scale=0.7]{MSP430_mem_small_dia.jpg}
\end{figure}
\begin{itemize}
\item Little-endian 
\item Untertilung in Byte und Word(2 Byte)
\end{itemize}
\end{frame}
\subsection{Mapping}
\begin{frame}
\includegraphics[scale=0.8,center]{MSP430_mem_big_dia.jpg}
\end{frame}
\section{CPU}
\subsection{Eigenschaften}
\begin{frame}
\ \\
\begin{itemize}
\item 16MHz/8MHz max. ( je nach Ausführung)
\item keine Minimalfrequenz
\item 16 16-Bit-Register ( 4 Spezielle, 12 Freie )
\item 7 unterschiedliche Adressmodi
\end{itemize}
Register:
\begin{itemize}
\item Program counter ( PC )
\item Stack pointer ( SP )
\begin{itemize}
\item speichert Rücksprungadresse
\item wächst abwärts oder schrumpft aufwärts
\end{itemize}
\item Status register ( SR )
\begin{itemize}
\item zeigt das Ergebnis der letzten Operation (C,Z,N,V)
\item steuert ebenfalls bestimmte MCU Modi.
\end{itemize}
\item Constant generator
\begin{itemize}
\item liefert durch Kombination von R2/R3 und den Adressmodi Standardwerte
\end{itemize}
\end{itemize}
\end{frame}
\subsection{Registergrafik}
\begin{frame}
\includegraphics[scale=0.45,center]{MSP430_alu_dia.jpg}
\end{frame}
\subsection{Adressierungsarten}
\begin{frame}
\begin{figure}[t]
\includegraphics[scale=0.7,center]{MSP430_instruktion_mode_dia.jpg}
\end{figure}
\begin{tabular}{p{0.5\textwidth}|p{0.5\textwidth}}
\begin{itemize}
\item R2 $\rightarrow$ Register-Mode
\item 0(R2) $\rightarrow$ Adress-Mode 
\item @R2 $\rightarrow$ 4
\item @R2+ $\rightarrow$ 8
\end{itemize}
&
\begin{itemize}
\item    R3 $\rightarrow$ 0
\item 0(R3) $\rightarrow$ 1
\item @R3 $\rightarrow$ 2
\item @R3+ $\rightarrow$ -1
\end{itemize}
\end{tabular}
\end{frame}
\subsection{Instruktionen}
\begin{frame}
Aufbau:
\begin{itemize}
\item 2 Operanden
\item 1 Operand
\item Sprünge
\end{itemize}
Merkmale:
\begin{itemize}
\item Byte und Word Adressierung
\item 16-Bit Instruktionslänge
\item 27 Standard-Befehle
\item trotz RISC: weitere 24 emulierte Befehle ( z.B.: CLR, INC, DEC, DECD, NOP) 
\item emulierte Befehle haben keine Nachteile
\end{itemize}
\end{frame}
\section{Clock}
\begin{frame}
\subsection{Clock-Typen}
2 Clock-Typen:
\begin{enumerate}
\item Niedrig-Frequenz-Clock
\begin{itemize}
\item meist ein Uhrenquarz mit 32KHz
\item dient zur regelmäßigen Steuerung der CPU
\item verbraucht im Dauerbetrieb extrem wenig Strom
\item dient als RTC
\end{itemize}
\item digital Steuerbarer Oszillator
\begin{itemize}
\item startet in weniger als $1\mu s$
\item dient zum schnellen "wecken" der MCU aus dem Low-Power-Mode
\item stellt eine Hochfrequenzuhr dar
\end{itemize}
\end{enumerate}
\end{frame}
\subsection{Clock-Grafik}
\begin{frame}
\includegraphics[scale=1,center]{MSP430_clock_dia.jpg}
\end{frame}
\section{Peripherie}
\subsection{Multplikator}
\begin{frame}
\includegraphics[scale=0.64,center]{MSP430_mult_dia.jpg}
\end{frame}
\end{document}